\documentclass[12pt]{article}
\usepackage[top=1in, bottom=1in, left=1in, right=1in]{geometry}

\usepackage{setspace}
\onehalfspacing

\newif\ifeqns
\eqnstrue

\usepackage[hang,flushmargin]{footmisc} 
% 'hang' flushes the footnote marker to the left,  'flushmargin'  flushes the text as well.

\def\baselinestretch{1}
\setlength{\parindent}{0mm} \setlength{\parskip}{0.8em}

\newlength{\up}
\setlength{\up}{-4mm}

\newlength{\hup}
\setlength{\hup}{-2mm}

\usepackage{amssymb}
%% The amsthm package provides extended theorem environments
\usepackage{amsthm}
\usepackage{epsfig}
\usepackage{times}
\renewcommand{\ttdefault}{cmtt}
\usepackage{amsmath}
\usepackage{graphicx} % for graphics files

% Draw figures yourself
\usepackage{tikz} 

% The float package HAS to load before hyperref
\usepackage{float} % for psuedocode formatting
\usepackage{xspace}

% from Denovo Methods Manual
\usepackage{mathrsfs}
\usepackage[mathcal]{euscript}
\usepackage{color}
\usepackage{array}

\usepackage[pdftex]{hyperref}
\usepackage{cancel}

\newcommand{\nth}{n\ensuremath{^{\text{th}}} }
\newcommand{\ve}[1]{\ensuremath{\mathbf{#1}}}
\newcommand{\Macro}{\ensuremath{\Sigma}}
\newcommand{\vOmega}{\ensuremath{\hat{\Omega}}}

\begin{document}
\begin{center}
{\bf NE 250, F17 \\
Transport Equation\\
August 25 \& September 1, 2017}
\end{center}

\setlength{\unitlength}{1in}
\begin{picture}(6,.1) 
\put(0,0) {\line(1,0){6.25}}         
\end{picture}

%-------------------------------------------------------------
\section*{Transport Equation}

Largely from Lewis and Miller Chp.\ 1 \cite{Lewis1993} and Duderstadt and Hamilton Chp.\ 4 \cite{Duderstadt1976}. Note: Duderstadt and Martin \cite{Duderstadt1979} is a very good general reference. It goes through all of this same stuff, but from a slightly more generic point of view (since this applies to \underline{any} collection of neutral particles).

\subsection*{Definitions}

\begin{figure}[h!]
    \begin{center}
    \includegraphics[keepaspectratio, width = 2.7 in]{../figs/phase-space}
    \end{center}
    \caption{Schematic of Phase Space}
    \label{fig:phase_space}
\end{figure}

Spatial logistics
\begin{itemize}
\item $d\vec{r} = d^3r$ = ordinary volume = $r^2 \sin(\theta) d\theta d\varphi dr$
%
\item $v$ = speed (scaler)
\item $\vec{v}$ = velocity (vector)
\item $d\vec{v} = d^3v$ = velocity volume = $v^2 \sin(\theta')d\theta' d\varphi' dr$
\item $v = \sqrt{(2E)/m}$ where $m$ is the rest mass of the particle. Thus, we can relate energy and speed.

\item $\vOmega$: unit directional vector in velocity space, $\vec{v} = v\vOmega$
\item $d\vOmega = \sin(\theta')d\theta' d\varphi' =  d^2\Omega$
%\item thus $d\vec{v} = v^2 dv d\vOmega$
\end{itemize}

These are the possible reactions we're generally going to worry about:

\hspace*{1em}total (t): all interactions. We can break total into:
\begin{itemize}
\item scattering (s): a neutron interacts with an atom and bounces off either elastically or inelastically, possibly causing it to change direction, energy, or both.
\item absorption (a): a neutron is absorbed by a nucleus. If this happens it might
\item fission (f): cause the nucleus to split into two pieces, releasing more neutrons.
\end{itemize}

Physics terms we will use:
\begin{enumerate}
\item \textbf{microscopic x-sec} ($\sigma$, [$cm^2$]): measure of the probability that an incident neutron will collide with a specific nucleus; $\sigma_j$ indicates a specific reaction, e.g.\ $j=f$ is fission.

\item \textbf{macroscopic x-sec} ($\Sigma$ [$cm^{-1}$]): measure of the probability per unit path length that an incident neutron will collide with a target
\[\Sigma_j = \sigma_j N\:,\]
where N is the atomic density of the target.

\item \textbf{double-differential scattering x-sec} ($\sigma_s(E, \vOmega \rightarrow E', \vOmega')dE' d\vOmega'$): measure of the probability that a neutron of energy $E$ and moving in direction $\vOmega$ scatters off of a specific nucleus into energy range $[E', E' + dE']$ and direction range $[\vOmega', \vOmega' + d\vOmega']$.

\item \textbf{fission yield} ($\nu(E)$): average \# of neutrons released by a fission induced by a neutron of energy E.

\item \textbf{fission spectrum} ($\chi(E)dE$): average \# of neutrons produced from fission that are born with energy in $[E, E + dE]$. This is normalized such that
\[\int_0^{\infty} \chi(E)dE =1\:.\]

\item \textbf{particle angular density} ($n(\vec{r}, E, \vOmega, t)d\vec{r} d\vOmega dE$): expected number of particles in volume element $d^3r$ at $\vec{r}$ whose energies are in $[E, E + dE]$ and direction of motion is in $[\vOmega, \vOmega + d\vOmega]$ at time $t$.

Note:
\ifeqns
\begin{align*}
n(\vec{r}, E, \vOmega, t) &= \frac{1}{mv}n(\vec{r}, v, \vOmega, t) \\
n(\vec{r}, v, \vOmega, t) &= v^2 n(\vec{r}, \vec{v}, t) \\
n(\vec{r}, \vec{v}, t) &= \frac{m}{v}n(\vec{r}, E, \vOmega, t)
\end{align*}
\else
\vspace*{3em}
\fi

\item \textbf{particle density}: ($N(\vec{r},E,t)d^3r dE$): expected number of particles in $d^3r$ at $\vec{r}$ whose energies are in $[E, E + dE]$ at time $t$.
\ifeqns
\[N(\vec{r},E,t)d^3r dE = \int_{4\pi} d\vOmega\: n(\vec{r}, E, \vOmega, t)d^3r dE \]
\else
\vspace*{2em}
\fi

\item \textbf{angular flux}: $\psi(\vec{r}, E, \vOmega, t) \equiv v n(\vec{r}, E, \vOmega, t)$.

\item \textbf{scalar flux}: $\phi(\vec{r},E,t) \equiv v N(\vec{r},E,t)$.
%
\ifeqns
\[= \int_{4\pi} d\vOmega\: \psi(\vec{r}, E, \vOmega, t) \]
\else
\vspace*{2em}
\fi

\item \textbf{interaction rate density}: expected number of $j$ reactions per volume per energy at time $t$.
%
\ifeqns
\[\int_{4\pi} d\vOmega \:\Sigma_j v n(\vec{r}, E, \vOmega, t) = \Sigma_j \phi(\vec{r},E,t)\]
\else
\vspace*{2em}
\fi

\item \textbf{angular current density} or partial current: $\vec{j}(\vec{r}, E, \vOmega, t) = \vec{v} n(\vec{r}, E, \vOmega, t)$; 

$\vec{j}(\vec{r}, E, \vOmega, t) \cdot \hat{e}\: dA\: dE\: d\vOmega$ is the expected number of particles crossing $dA$ along unit direction $\hat{e}$ with energy in $[E, E + dE]$ and direction in $[\vOmega, \vOmega + d\vOmega]$ at time $t$.

\item \textbf{net current}: $\vec{J}(\vec{r}, E, t) $ is the net \# of particles crossing a unit area per second along a direction normal to that area with energies in $[E, E + dE]$ at time $t$.
\ifeqns
\[\vec{J}(\vec{r}, E, t) = \int_{4\pi} d\vOmega\: \vOmega \psi(\vec{r}, E, \vOmega, t)\]
\else
\vspace*{2em}
\fi

\end{enumerate}

%-----------------------------------------
\subsection*{Assumptions}
\begin{enumerate}
\item Particles are point objects ($\lambda = h/(mv))$ is small compared to the atomic diameter): its state is fully described by its location, velocity vector, and a given time. This ignores rotation and quantum effects.

\item Neutral particles travel in straight lines between collisions.

\item Particle-particle collisions are negligible (makes TE linear).

\item Material properties are isotropic (generally valid unless velocities are very low).

\item Material composition is time-independent (generally valid over short time scales).

\item Quantities are expected values: fluctuations about the mean for very low densities are not accounted for.
\end{enumerate}


%-----------------------------------------
\section*{Derivation}
The TE is a \underline{detailed} balance of the particle population over phase space that is as close to exact as possible. 

DRAW a Differential volume picture.

Consider a volume $V$ with surface $S$. For each point $\vec{r} \in S$, let $\hat{e}_S$ be the outward normal vector.

For a given $\vOmega$, define $S^+$ as that part of $S$ for which $\hat{e}_S \cdot \vOmega > 0$ (outgoing particles) and $S^-$ as that part of $S$ for which $\hat{e}_S \cdot \vOmega < 0$ (incoming particles).

Then, for this volume $V$ for a fixed $E$ and $\vOmega$, the general rate equation can be written for particles satisfying $\vec{r} \in V$, energies in $[E, E+dE]$ and direction $[\vOmega, \vOmega + d\vOmega]$ as:

\hspace*{3 em} \boxed{\text{Rate of change of the particle (neutron) population 
 = rate of production - rate of loss}}

%--------------
\subsection*{Rate of Change}
Recall the definition of $n$: expected number of particles in volume element $d^3r$ at $\vec{r}$ whose energies are in $[E, E + dE]$ and direction of motion is in $[\vOmega, \vOmega + d\vOmega]$ at time $t$.

To get the rate of change of particles within the volume, we need to integrate over volume and take the derivative with respect to time:
%
\ifeqns
\[\boxed{\Bigl[ \int_V \frac{\partial}{\partial t} n(\vec{r}, E, \vOmega, t) d\vec{r} \Bigr] dE d\vOmega }\]
\else
\vspace*{2.5em}
\fi

%--------------
\subsection*{Production Mechanisms}
How can we produce neutrons in volume element $d^3r$ at $\vec{r}$ whose energies are in $[E, E + dE]$ and direction of motion is in $[\vOmega, \vOmega + d\vOmega]$ at time $t$?
%
\begin{enumerate}
\item Inscattering (from some other energy and/or angle into our energy and angle),
\item Fission neutrons, or
\item Fixed/interior sources.
\end{enumerate}

1) Scattering into $[E, E + dE]$ and $[\vOmega, \vOmega + d\vOmega]$

This is the definition of the double differential scattering cross section:
\ifeqns
\[\boxed{\Bigl[\int_V d^3r \int_{4\pi} d\vOmega' \int_0^{\infty} dE' \: \Sigma_s(E', \vOmega' \rightarrow E, \vOmega) v' n(\vec{r}, E', \vOmega', t) \Bigr] dE d\vOmega}\]
\else
\vspace*{3em}
\fi

2) Expected rate of neutron production by fission

Note: fission neutrons are isotropic, thus they are produced at $\frac{1}{4\pi}$ per steradian. This means the fraction within $[\vOmega, \vOmega + d\vOmega]$ is $\frac{d\vOmega}{4\pi}$ (fyi: including this normalization differs among textbooks and research papers).

Also, recall that $\chi(E)dE$ is the fraction of neutrons born into $[E, E + dE]$. Thus
%
\ifeqns
\[\boxed{\frac{\chi(E)}{4\pi}\Bigl[\int_V d^3r \int_{4\pi} d\vOmega' \int_0^{\infty} dE' \: \nu(E') \Sigma_f(E') v' n(\vec{r}, E', \vOmega', t) \Bigr] dE d\vOmega}\]
\else
\vspace*{3em}
\fi

3) Production from a fixed source

Sources are fully specified by a function reminiscent of the $n$ definition: $s(\vec{r}, E, \vOmega, t)$ s.t.\ \\$s(\vec{r}, E, \vOmega, t)d^3rdEd\vOmega \equiv$ the expected number of particles that are produced at time $t$ inside volume $d^3r$ at $\vec{r}$ with energy $[E, E + dE]$ and direction $[\vOmega, \vOmega + d\vOmega]$.

Rate of production of such particles in $V$ is
\ifeqns
\[\boxed{\Bigl[\int_V d^3r \:s(\vec{r}, E, \vOmega, t) \Bigr] dE d\vOmega }\]
\else
\vspace*{2.5em}
\fi

%--------------
\subsection*{Loss Mechanisms}
How can we lose neutrons from volume element $d^3r$ at $\vec{r}$ whose energies are in $[E, E + dE]$ and direction of motion is in $[\vOmega, \vOmega + d\vOmega]$ at time $t$?
%
\begin{enumerate}
\setcounter{enumi}{3}
\item Neutrons can collide and exit the phase space (any collision will change its state) or
\item Neutrons can stream into other locations and/or directions of motion (leakage).
\end{enumerate}

4) Total interaction: any collision can lead to (a) absorption or (b) a change in $E$ or $\vOmega$ or both.
\ifeqns
\[\boxed{\Bigl[\int_V d^3r \Sigma_t(E) v n(\vec{r}, E, \vOmega, t) \Bigr] dE d\vOmega }\]
\else
\vspace*{3em}
\fi

5) Net leakage out of phase space

\begin{figure}[h!]
\begin{center}
%\includegraphics[height=1in]{UnitXsecArea}
\includegraphics[height=1in]{../figs/DifferentialArea}
\end{center}
\end{figure}

Use the definition of $\vec{j}$: the expected number of particles crossing $dS$ along $\hat{e}_S$ with energy $[E, E + dE]$ and direction $[\vOmega, \vOmega + d\vOmega]$ at time $t$  
%
\[= \vec{j}(\vec{r}_s, E, \vOmega, t) \cdot d\vec{S} dE d\vOmega\:,\]
%
where $\vec{r}_s$ is a point on the surface and $d\vec{S} = \hat{e}_S dS$.

Thus, the total leakage out of $V$ is
\ifeqns
\[\Bigl[\int_S \vec{j}(\vec{r}_s, E, \vOmega, t) \cdot d\vec{S} \Bigr] dE d\vOmega\:. \]
\else
\vspace*{3em}
\fi

We can use divergence theorem to rewrite this w.r.t.\ $V$ (rather than $S$):
\ifeqns
\[\int_S \hat{e}_S \cdot \vec{F} (\vec{r}) dS = \int_V \nabla \cdot \vec{F} (\vec{r}) dV\]
\else
\vspace*{3em}
\fi
%where
%\[\nabla = \vec{i}\frac{d}{dx} + \vec{j}\frac{d}{dy} + \vec{k}\frac{d}{dz}\]
%is the gradient operator.

This gives
\ifeqns
\[\Bigl[\int_V d^3r \: \nabla \cdot \bigl( \underbrace{\vec{j}(\vec{r}_s, E, \vOmega, t)}_{=\vec{v}n = v \vOmega n} \bigr) \Bigr] dE d\vOmega \]
\else
\vspace*{3em}
\fi
%
And we can use this identity:
\ifeqns
\begin{align}
%
\vOmega \cdot (\nabla f) &= \nabla \cdot \vOmega f \text{ because }\vOmega\text{ is not a function.} \nonumber\\
\nabla \cdot \vOmega f &= f(\underbrace{\nabla \cdot \vOmega}_{0}) + \vOmega \cdot (\nabla f) \nonumber
\end{align}
%
\[\therefore \:\: \boxed{\Bigl[\int_V d^3r \: \vOmega \cdot \nabla \bigl(v n(\vec{r}_s, E, \vOmega, t) \bigr) \Bigr] dE d\vOmega }\]
\else
\vspace*{5em}
\fi

Note: $\vOmega \cdot \nabla$ represents the derivative along the direction of motion.

%--------------
\subsection*{All Together Now}
The balance of neutrons: rate of change - production + loss = 0.

Suppressing dependencies to save space for the moment
%
\begin{align}
\int_V d^3r \Bigl[\frac{\partial n}{\partial t} &- 
\int_{4\pi} d\vOmega' \int_0^{\infty} dE' \:\Sigma_s(E', \vOmega' \rightarrow E, \vOmega) v' n' \nonumber\\&-
\frac{\chi(E)}{4\pi} \int_{4\pi} d\vOmega' \int_0^{\infty} dE' \:\nu(E') \Sigma_f(E') v' n' -
s +
\Sigma_t v n + 
\vOmega \cdot \nabla v n \Bigr] = 0 \nonumber
\end{align}

We note that since the volume was arbitrarily chosen, the integral will only vanish if the integrand is zero 
\[\int_{\text{any }V} d^3r f(\vec{r}) = 0 \rightarrow f(\vec{r}) = 0 \:.\]

Now we have a balance relation that we can rearrange, substituting $\psi(\vec{r}, E, \vOmega, t) = vn(\vec{r}, E, \vOmega, t)$, to get what we usually call the Boltzmann Equation for neutron transport
%
\begin{align}
\frac{1}{v}\frac{\partial \psi(\vec{r}, E, \vOmega, t)}{\partial t} &+ 
\vOmega \cdot \nabla \psi(\vec{r}, E, \vOmega, t) +
\Sigma_t \psi(\vec{r}, E, \vOmega, t) = \nonumber\\
%
& \int_{4\pi} d\vOmega' \int_0^{\infty} dE' \Sigma_s(E', \vOmega' \rightarrow E, \vOmega) \psi(\vec{r}, E', \vOmega', t)  +\nonumber\\
%
& \frac{\chi(E)}{4\pi} \int_0^{\infty} dE' \nu(E') \Sigma_f(E') \int_{4\pi} d\vOmega' \psi(\vec{r}, E', \vOmega', t) +
s(\vec{r}, E, \vOmega, t) \nonumber
\end{align}

%--------------
\subsection*{Initial and Boundary Conditions}
\begin{enumerate}
\item \underline{Initial Condition}: we start with some initial ``known" state:
\[\psi(\vec{r}, E, \vOmega, 0) = \psi_0(\vec{r}, E, \vOmega)\]
for the problem domain. Note, the initial flux can be a functional expression.

\item \underline{Interface Condition}: the angular flux must be continuous along $\vOmega$ at all points, including material interfaces.

\begin{minipage}{0.45\textwidth}
\begin{figure}[H]
\includegraphics[height=2in]{../figs/InterfaceCondition}
%\caption{\label{fig:interfaceCondition} Interface Condition}
\end{figure}
\end{minipage} \hfill
\begin{minipage}{0.45\textwidth}
\ifeqns
$\psi_1(\vec{r}_S, E, \vOmega, t) = \psi_2(\vec{r}_S, E, \vOmega, t)$

\vspace*{1 em}
$\forall E$ and $\vOmega$.
\else
\vspace*{3em}
\fi
\end{minipage}

\item \underline{Fixed Condition}: you can specify incoming flux

\begin{minipage}{0.45\textwidth}
\begin{figure}[H]
\includegraphics[height=1.75in]{../figs/FreeSurfaceCondition}
%\caption{\label{fig:FixedCondition} Fixed Condition}
\end{figure}
\end{minipage} \hfill
\begin{minipage}{0.45\textwidth}
$\psi(\vec{r}_S, E, \vOmega, t) = \psi_{IN}(\vec{r}_S, E, \vOmega, t)$ for \\$\vec{e} \cdot \vOmega < 0$: specifying incoming neutrons. Note, the incoming flux can be a functional expression; it can also be zero.

This is equivalent to specifying the incoming partial current, $$\vec{j}^-(\vec{r}_S, E, t) = \int_{\vec{e} \cdot \vOmega < 0} d\vOmega \:(\vec{e} \cdot \vOmega) \psi(\vec{r}_S, E, \vOmega, t)\:.$$
\end{minipage}

\item \underline{Reflective Condition}: there is mirror symmetry at some surface:
\[\psi(\vOmega_{IN}) = \psi(\vOmega_{OUT})\:.\]

\item \underline{Periodic Condition}: you know there is a repetition in the system
\ifeqns
\[\psi(\vec{r}_S, E, \vOmega, t) = \psi(\vec{r}_S \pm \vec{p}, E, \vOmega, t)\:.\]
\else
\vspace*{3em}
\fi

\item \underline{Finiteness Condition}: to by physically valid we need to meet the condition \\$0 \leq \psi(\vec{r}, E, \vOmega, t) < \infty$, with the possible exception of point sources\dots

\item which we handle with the \underline{Source Condition}: localized sources are introduced as mathematical singularities at the location of the source.

For a source $s(\vec{r}_0, E, \vOmega, t)$:
\ifeqns
\begin{align*}
&\lim_{\vec{r} \rightarrow \vec{r}_0} \int_S dS\: \vec{e} \cdot \vOmega \psi(\vec{r}, E, \vOmega, t) = s(\vec{r}_0, E, \vOmega, t)\:, \\
&s(\vec{r}, E, \vOmega, t) = s(\vec{r}_0, E, \vOmega, t)\delta(\vec{r} - \vec{r}_0)\:.
\end{align*}
\else
\vspace*{5em}
\fi
\end{enumerate}


%---------------------------------------------
%---------------------------------------------
\section*{Simplified Forms}

\subsection*{One Speed}
Assume all particles are at the same speed, so we no longer need energy dependence. To do this, we assume all particles have the same speed, $\vec{v} = v_0 \cdot \vOmega$. Thus
%%
%\begin{align}
%n(\vec{r}, v, \vOmega, t) &= n(\vec{r}, \vOmega, t) \delta(v - v_0) \\
%\Sigma_s(\underbrace{E' \rightarrow E, \vOmega' \cdot \vOmega}_{\text{another way to write this}}) &= \Sigma_s(E,\vOmega' \cdot \vOmega)\delta(E' - E)
%\end{align}
%
% remove E integration and E dependence

\begin{align*}
\frac{1}{v}\frac{\partial \psi(\vec{r}, \vOmega, t)}{\partial t} &+ 
\vOmega \cdot \nabla \psi(\vec{r}, \vOmega, t) +
\Sigma_t \psi(\vec{r}, \vOmega, t) = \nonumber\\
%
& \int_{4\pi} d\vOmega' \Sigma_s(\vOmega' \rightarrow \vOmega) \psi(\vec{r}, \vOmega', t)  
+ \frac{\nu \Sigma_f}{4\pi} \int_{4\pi} d\vOmega' \psi(\vec{r},  \vOmega', t) 
+ s(\vec{r}, \vOmega, t) 
%Q(\vec{r}, \vOmega, t)
\end{align*}
%
%Where:
%\begin{align}
%Q(\vec{r}, \vOmega, t) &= \frac{1}{4\pi} \nu \Sigma_f \int_{4\pi} d\vOmega' \psi(\vec{r},  \vOmega, t) + S(\vec{r}, \vOmega, t) \\
%&= \frac{1}{4\pi} \bigl( \nu \Sigma_f \phi(\vec{r}, t) + s(\vec{r}, t) \bigr)
%\end{align}


%---------------------------------------------
\subsection*{One Speed, One Dimensional}

Another common simplification that can enable analytical solutions is to also reduce to one dimension (which can be done retaining energy dependence, we simply do not do that here), so we'll get rid of $y$ and $z$.

%\begin{minipage}{0.5\textwidth}
$\vec{r} = (x, y, z)$

$d\vOmega = \sin(\theta) d\theta	d\varphi = d\mu d\varphi$

Note $\mu = \cos(\theta)$ so $d\mu = \sin(\theta)$

$\Omega_x = \cos(\theta) = \mu$
%\end{minipage} \hfill
%\begin{minipage}{0.45\textwidth}
%\begin{figure}[H]
%\includegraphics[height=1in]{1Dspace}
%\caption{swap z for y, y for x, and x for z}
%\end{figure}
%\end{minipage}

% \vec{r} switched to x
\begin{align*}
\frac{1}{v}\frac{\partial \psi(x, \vOmega, t)}{\partial t} &+ 
\bigl(\Omega_x \frac{\partial}{\partial x} + \cancel{\Omega_y \frac{\partial}{\partial y}} + \cancel{\Omega_z \frac{\partial}{\partial z}} \bigr)\psi(x, \vOmega, t) +
\Sigma_t \psi(x, \vOmega, t) = \nonumber\\
%
& \int_{4\pi} d\vOmega' \Sigma_s(\vOmega' \rightarrow \vOmega) \psi(x, \vOmega', t)  + 
\frac{\nu \Sigma_f}{4\pi} \int_{4\pi} d\vOmega' \psi(x,  \vOmega', t) + s(x, \vOmega, t)%+ Q(x, \vOmega, t) \nonumber
\end{align*}


%\begin{minipage}{0.3\textwidth}
%\begin{figure}[H]
%\includegraphics[height=1.5in]{AzimuthalSymmetry}
%%\caption{\label{fig:FixedCondition} Fixed Condition}
%\end{figure}
%\end{minipage} \hfill
%\begin{minipage}{0.65\textwidth}
%
%\end{minipage}
%
%Using this idea we can write things more cleanly:
%%
%% switch to just \mu
%\begin{align*}
%\frac{1}{v_0}\frac{\partial \psi(x, \vOmega, t)}{\partial t} &+ 
%\mu \frac{\partial}{\partial x}\psi(x, \vOmega, t) +
%\Sigma_t \psi(x, \vOmega, t) = \nonumber\\
%%
%& \int_0^{2\pi} d\phi \int_{4\pi} d\vOmega' \Sigma_s(\vOmega' \cdot \vOmega) \frac{\psi(x, \mu', t)}{2\pi}  + 
%\frac{1}{4\pi} \nu \Sigma_f \int_0^{2\pi} d\phi \psi(x, \mu, t) + s(x, \mu, t)% 2\pi Q(x, \mu, t)
%\end{align*}
%
%If \textbf{scattering is also isotropic}:
%\[\Sigma_s(\vOmega' \cdot \vOmega) = \frac{\Sigma_s}{4\pi} \]
%
%And we get the 1-group, 1-D, isotropic neutron TE:
%%
%% scattering is now iso, add in sources
%\begin{align*}
%\frac{1}{v_0}\frac{\partial \psi(x, \mu, t)}{\partial t} &+ 
%\mu \frac{\partial}{\partial x}\psi(x, \mu, t) +
%\Sigma_t \psi(x, \mu, t) = \nonumber\\
%%
%& \frac{1}{2} \Sigma_s \int_{-1}^{1} d\mu' \psi(x, \mu', t)  
%+  \frac{1}{2} \bigl( \nu \Sigma_f \phi(x, t) + S(x, t) \bigr)
%\end{align*}
%
%%---------------------------------------------
%\subsection*{Steady State}
%If we get rid of time dependence.
%%
%\begin{equation}
%\mu \frac{\partial}{\partial x}\psi(x, \mu) +
%\Sigma_t \psi(x, \mu) = \frac{1}{2} \Sigma_s \int_{-1}^{1} d\mu' \psi(x, \mu')  
%+  \frac{1}{2} \bigl( \nu \Sigma_f \phi(x) + S(x) \bigr) \nonumber
%\end{equation}
%
%
%And very finally - we can have the unrealistic case of \textbf{purely absorbing media}:
%%
%\begin{equation}
%\mu \frac{\partial}{\partial x}\psi(x, \mu) +
%\Sigma_a \psi(x, \mu) = \frac{1}{2} \bigl( \nu \Sigma_f \phi(x) + S(x) \bigr) \nonumber
%\end{equation}


%--------------------------------------------
%--------------------------------------------
%--------------------------------------------


%--------------------------------------------------------------------
\bibliographystyle{plain}
\bibliography{te-de} 

\end{document}
