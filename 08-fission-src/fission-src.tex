\documentclass[12pt]{article}
\usepackage[top=1in, bottom=1in, left=1in, right=1in]{geometry}

\usepackage{setspace}
\onehalfspacing

\newif\ifeqns
\eqnstrue

\usepackage{amssymb}
%% The amsthm package provides extended theorem environments
\usepackage{amsthm}
\usepackage{epsfig}
\usepackage{times}
\renewcommand{\ttdefault}{cmtt}
\usepackage{amsmath}
\usepackage{graphicx} % for graphics files

% Draw figures yourself
\usepackage{tikz} 

% writing elements
\usepackage{mhchem}

% The float package HAS to load before hyperref
\usepackage{float} % for psuedocode formatting
\usepackage{xspace}

% from Denovo Methods Manual
\usepackage{mathrsfs}
\usepackage[mathcal]{euscript}
\usepackage{color}
\usepackage{array}

\usepackage[pdftex]{hyperref}
\usepackage[parfill]{parskip}

% math syntax
\newcommand{\nth}{n\ensuremath{^{\text{th}}} }
\newcommand{\ve}[1]{\ensuremath{\mathbf{#1}}}
\newcommand{\Macro}{\ensuremath{\Sigma}}
\newcommand{\rvec}{\ensuremath{\vec{r}}}
\newcommand{\vecr}{\ensuremath{\vec{r}}}
\newcommand{\omvec}{\ensuremath{\hat{\Omega}}}
\newcommand{\vOmega}{\ensuremath{\hat{\Omega}}}
\newcommand{\sigs}{\ensuremath{\Sigma_s(\rvec,E'\rightarrow E,\omvec'\rightarrow\omvec)}}
\newcommand{\el}{\ensuremath{\ell}}
\newcommand{\sigso}{\ensuremath{\Sigma_{s,0}}}
\newcommand{\sigsi}{\ensuremath{\Sigma_{s,1}}}
\newcommand{\cc}[1]{\ensuremath{\overline{#1}}}
\newcommand{\ccm}[1]{\ensuremath{\overline{\mathbf{#1}}}}
%---------------------------------------------------------------------------
%---------------------------------------------------------------------------
\begin{document}
\begin{center}
{\bf NE 250, F17\\
October 6, 2017 
}
\end{center}

Duderstadt and Hamilton Chp.\ 5.

So far we have looked at fixed source diffusion problems and methods for solving them analytically (note: numerical solution is the focus of 155 for DE and 255 for TE). Frequently, we care about systems with fission. We'll focus on that next.

We'll take a moment to talk through what's happening with fission in reactors. Neutrons are born at ``high" energies, in the MeV range. 
\begin{itemize}
\item \textbf{Thermal}: neutrons can cause fission at fast energies in any of $^{235}$U, $^{239}$Pu, or $^{238}$U. However, it is much more likely neutrons will scatter and lose energy (be moderated) by nuclei such as H or C. As the neutrons are slowed, they pass through absorption resonances of heavy  materials, especially $^{238}$U. Neutrons could also leak out of the system during that slowing down process. In thermal systems most ($\sim 85\%-90\%$) neutrons make it to thermal energies. There they will diffuse about the core and leak or be absorbed. Some of those that are absorbed will be absorbed in fuel, and some will cause fission.

\item \textbf{Fast}: In fast reactors we try to keep the neutrons fast. Leakage will still play a role at fast energies. Resonance absorption and downscattering will be much less. 
\end{itemize}
%
It is clear that energy is quite important in the fission process. Fundamentally, we want to know the fission neutrons that are born in the energy range $[E, E+dE]$ into angle $d\vOmega$ and $\vOmega$. This gives a fission term of 
\[
\frac{\chi(E)}{4\pi} \int_0^{\infty} dE' \int_{4\pi} d\vOmega' \: \nu(E') \Sigma_f(\vec{r}, E') \psi(\vec{r}, E', \vOmega', t) \:.  
\] 
In diffusion we use the scalar flux rather than the angular flux. Further, for our current work we'll collapse to the one-group case to get an effective fission source:
\[
\langle \nu \Sigma_f (\vec{r}) \rangle = \frac{\int_0^{\infty} dE'\: \nu(E') \Sigma_f(\vec{r}, E') \phi(\vec{r}, E', t)} {\int_0^{\infty} dE'\: \phi(\vec{r}, E', t)} \equiv \nu \Sigma_f (\vec{r}) 
\]

We can write the one-group diffusion equation with fission
\begin{equation}
\frac{1}{v} \frac{\partial \phi(\vec{r}, t)}{\partial t} - \nabla \cdot D(\vec{r}) \nabla	\phi(\vec{r}, t) + \Sigma_a(\vec{r}) \phi(\vec{r}, t) = \Sigma_f (\vec{r})\phi(\vec{r}, t)
\end{equation}

\subsection*{Solutions}
Let's look at some example cases. 

\end{document}
