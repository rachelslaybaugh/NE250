\documentclass[12pt]{article}
\usepackage[top=1in, bottom=1in, left=1in, right=1in]{geometry}

\usepackage{setspace}
\onehalfspacing

\usepackage{amssymb}
%% The amsthm package provides extended theorem environments
\usepackage{amsthm}
\usepackage{epsfig}
\usepackage{times}
\renewcommand{\ttdefault}{cmtt}
\usepackage{amsmath}
\usepackage{graphicx} % for graphics files
\usepackage{tabu}

% Draw figures yourself
\usepackage{tikz} 

% writing elements
\usepackage{mhchem}

% The float package HAS to load before hyperref
\usepackage{float} % for psuedocode formatting
\usepackage{xspace}

% from Denovo Methods Manual
\usepackage{mathrsfs}
\usepackage[mathcal]{euscript}
\usepackage{color}
\usepackage{array}

\usepackage[pdftex]{hyperref}
\usepackage[parfill]{parskip}

% math syntax
\newcommand{\nth}{n\ensuremath{^{\text{th}}} }
\newcommand{\ve}[1]{\ensuremath{\mathbf{#1}}}
\newcommand{\Macro}{\ensuremath{\Sigma}}
\newcommand{\rvec}{\ensuremath{\vec{r}}}
\newcommand{\vecr}{\ensuremath{\vec{r}}}
\newcommand{\omvec}{\ensuremath{\hat{\Omega}}}
\newcommand{\vOmega}{\ensuremath{\hat{\Omega}}}
\newcommand{\sigs}{\ensuremath{\Sigma_s(\rvec,E'\rightarrow E,\omvec'\rightarrow\omvec)}}
\newcommand{\el}{\ensuremath{\ell}}
\newcommand{\sigso}{\ensuremath{\Sigma_{s,0}}}
\newcommand{\sigsi}{\ensuremath{\Sigma_{s,1}}}
\newcommand{\ep}{\ensuremath{\varepsilon}}
%---------------------------------------------------------------------------
%---------------------------------------------------------------------------
\begin{document}
\begin{center}
{\bf NE 250, F15\\
November 18, 2015 
}
\end{center}

We've talked about how to discretize energy using the multigroup approximation and angle using $S_N$, $P_N$, or $SP_N$ for the transport equation. \\
This results in a set of ODEs and PDEs with space as the only variable.\\
In this class we're going to talk about options for handling that.
\begin{itemize}
\item ray tracing: this is what we do in Monte Carlo and the integral form of the TE. This allows us to represent ``exact" geometry.
\item There are two ways we can get equation sets with $\nabla^s$ operators: 
  \begin{itemize}
  \item the Diffusion equation. We can apply standard methods for this (learned in 150 I think) or we can apply spherical harmonics in space, which can be manipulated to look a like a bunch of diffusion equations and we handle these as the regular DE.
  \item the even-odd parity TE (which we won't cover). This still has angular dependence but a diffusion-like operator so again we can use the same methods.
  \end{itemize}
\item We will look at the first order form ($\nabla$) here. 
\end{itemize}

\textbf{Spatial Discretization in Slab Geometry} (L\&M 3.3)\\
We'll start by thinking about the 1D, 1-group TE equation that has $S_N$ applied:
\begin{align*}
\mu_n \frac{d \psi_n}{dx} &+ \Sigma_t(x) \psi_n(x) = \sum_{l=0}^L (2l+1) P_l(\mu_n) \Sigma_{s,l}(x) \phi_l(x) + s_n(x)\\
\phi_l(x) &= \frac{1}{2}\sum_{n=1}^N w_n P_l(\mu_n) \psi_n(x)
\end{align*}

- discretization\\
- boundary conditions and sweep pattern\\
- local truncation error

\textbf{Spatial Discretization in Multi-D} (L\&M 4.3)\\

- discretization\\
- boundary conditions and sweeping  pattern\\
- quick mention of other spatial methods and their properties


rest of course (3 classes left):\\
- M, 11/23: equation solution procedures (inner iteration, outer iteration, eigenvalue iteration, convergence); one basic solver of each type (SI, GS, PI)\\ 
- M, 11/30: example of research into a different solver for each type (block Krylov, RQI)
- W, 12/02: choice preconditioners? computing architectures? some topic we skipped? Last half of class will be feedback and discussion of class.




\end{document}