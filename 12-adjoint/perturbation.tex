\documentclass[12pt]{article}
\usepackage[top=1in, bottom=1in, left=1in, right=1in]{geometry}

\usepackage{setspace}
\onehalfspacing

\usepackage{amssymb}
%% The amsthm package provides extended theorem environments
\usepackage{amsthm}
\usepackage{epsfig}
\usepackage{times}
\renewcommand{\ttdefault}{cmtt}
\usepackage{amsmath}
\usepackage{graphicx} % for graphics files

% Draw figures yourself
\usepackage{tikz} 

% writing elements
\usepackage{mhchem}

% The float package HAS to load before hyperref
\usepackage{float} % for psuedocode formatting
\usepackage{xspace}

% from Denovo Methods Manual
\usepackage{mathrsfs}
\usepackage[mathcal]{euscript}
\usepackage{color}
\usepackage{array}

\usepackage[pdftex]{hyperref}
\usepackage[parfill]{parskip}

% math syntax
\newcommand{\nth}{n\ensuremath{^{\text{th}}} }
\newcommand{\ve}[1]{\ensuremath{\mathbf{#1}}}
\newcommand{\Macro}{\ensuremath{\Sigma}}
\newcommand{\rvec}{\ensuremath{\vec{r}}}
\newcommand{\vecr}{\ensuremath{\vec{r}}}
\newcommand{\omvec}{\ensuremath{\hat{\Omega}}}
\newcommand{\vOmega}{\ensuremath{\hat{\Omega}}}
\newcommand{\sigs}{\ensuremath{\Sigma_s(\rvec,E'\rightarrow E,\omvec'\rightarrow\omvec)}}
\newcommand{\el}{\ensuremath{\ell}}
\newcommand{\sigso}{\ensuremath{\Sigma_{s,0}}}
\newcommand{\sigsi}{\ensuremath{\Sigma_{s,1}}}
%---------------------------------------------------------------------------
%---------------------------------------------------------------------------
\begin{document}
\begin{center}
{\bf NE 250, F17\\
October 25, 2017 
}
\end{center}

The Adjoint Equation in \textbf{Multiplying Media}

We will call the fission operator $G$, where
\[
G \zeta = \frac{\chi(E)}{4\pi}\int_0^{\infty} dE' \: \nu(E') \Sigma_f(E') \int_{4\pi} d\vOmega' \zeta(\rvec, \vOmega', E')
\]
To get the adjoint version, we can take the same approach as with the scattering operator to get
\[
G^{\dagger} \zeta^{\dagger} = \frac{\nu(E) \Sigma_f(E)}{4\pi}\int_0^{\infty} dE' \: \chi(E') \int_{4\pi} d\vOmega' \zeta^{\dagger}(\rvec, \vOmega', E')\:.
\]
Note that this operator obeys the adjoint identity as well: $\langle\zeta^{\dagger}, G \zeta\rangle = \langle\zeta, G^{\dagger} \zeta^{\dagger}\rangle$.

We can combine this operator with $H$ to use the techniques in the preceding section to look at subcritical systems.\\
We replace $H$ and $H^{\dagger}$ with $H-G$ and $H^{\dagger}-G^{\dagger}$, respectively.\\

To look at the \textit{eigenvalue} form of the problem:
\[
H\psi - \frac{1}{k}G\psi = 0
\]
where $\psi$ is the fundamental solution mode, $\psi > 0$ for $\rvec \in V$, $\psi(\vec{r}_s, \vOmega, E) = 0$ for $\hat{n} \cdot \vOmega < 0$ with $\vec{r}_s \in \Gamma$.  
\[
H^{\dagger}\psi^{\dagger} - \frac{1}{k^{\dagger}}G^{\dagger}\psi^{\dagger} = 0
\]
Now $\psi^{\dagger}$ is the fundamental mode, $\psi^{\dagger} > 0$ for $\rvec \in V$, $\psi^{\dagger}(\vec{r}_s, \vOmega, E) = 0$ for $\hat{n} \cdot \vOmega > 0$ with $\vec{r}_s \in \Gamma$.  

Next, compare the two equations the way we did for response: multiply the forward equation by $\psi^{\dagger}$, integrate both over volume, and subtract for the adjoint equation multiplied by $\psi$ and integrated:
\begin{align*}
\langle\psi^{\dagger}, H\psi\rangle - \langle\psi, H^{\dagger} \psi^{\dagger}\rangle &- \frac{1}{k}\langle\psi^{\dagger}, G\psi\rangle + \frac{1}{k^{\dagger}}\langle\psi, G^{\dagger}\psi^{\dagger}\rangle = 0\\
%
\text{and because }&\langle\psi^{\dagger}, G \psi\rangle = \langle\psi, G^{\dagger} \psi^{\dagger}\rangle\:,\\
%
\bigl(\frac{1}{k} -& \frac{1}{k^{\dagger}} \bigr)\langle\psi^{\dagger}, G\psi\rangle  = 0 \\
\therefore \: k &= k^{\dagger} \quad \text{because }G\psi > \:.
\end{align*}

Cool. What do we do with that? We can figure out the effect of small changes to critical systems, which is super useful in NE!

----------------------------\\
\textbf{Reactivity} of a \textbf{Perturbed System}

Recall that reactivity is the measure of a system's deviation from critical: $\rho = \frac{k-1}{k}$.\\
We are interested in how small changes (perturbations) in a system will cause reactivity to change.\\
We can use the adjoint equation to help us.

We will denote small changes with a preceding $\delta$ and unperturbed values with a subscript $0$. Thus, for small changes in $k$ we can define 
\[
\delta \rho = \frac{\delta k}{k_0^2}
\]
For an unperturbed reactor we have these fundamental mode equations:
\begin{align*}
H_0\psi_0 &- \frac{1}{k_0}G_0\psi_0 = 0\\
H_0^{\dagger}\psi_0^{\dagger} &- \frac{1}{k_0^{\dagger}}G_0^{\dagger}\psi_0^{\dagger} = 0
\end{align*}
We can represent system perturbation by adding small quantities to the operators and fluxes (very small compared to things themselves). For the forward system we get
\[
(H_0 + \delta H)(\psi_0 + \delta \psi) - \frac{1}{k_0 + \delta k}(G_0 + \delta G)(\psi_0 + \delta \psi) = 0
\]
We'll multiply the system out and, because perturbations are small, we can ignore second order terms (anything with two $\delta$s multiplied together). We then do some rearranging
\begin{align*}
H_0 \psi_0 + \delta H \psi_0 &+ H_0\delta\psi - \bigl(\frac{1}{k_0 + \delta k}\bigr)\bigl(G_0\psi_0 + \delta G \psi_0 + G_0\delta\psi \bigr) \approx 0 \\
%
\frac{1}{k_0 + \delta k} &= \frac{1}{k_0(1 + \delta k/k_0)} \approx \frac{1}{k_0}\bigl(1 - \frac{\delta k}{k_0}\bigr) \\
\\
%
-\frac{\delta k}{k_0^2}G_0\psi_0 &\approx \underbrace{\bigl(H_0 - \frac{1}{k_0}G_0 \bigr)\psi_0}_{\text{fundamental mode}=0} + \bigl(\delta H - \frac{1}{k_0}\delta G\bigr)\psi_0  + \bigl(H_0 - \frac{1}{k_0}G_0 \bigr)\delta \psi
\end{align*}
To set ourselves up for the next step, we again multiply by $\psi_0^{\dagger}$ and integrate over phase space to get
\begin{equation}
-\frac{\delta k}{k_0^2} \langle \psi_0^{\dagger}, G_0\psi_0 \rangle \approx \langle \psi_0^{\dagger},\bigl(\delta H - \frac{1}{k_0}\delta G\bigr)\psi_0 \rangle + \langle \psi_0^{\dagger},\bigl(H_0 - \frac{1}{k_0}G_0 \bigr)\delta \psi \rangle
\label{eq:perturb}
\end{equation}

Next, we can use the adjoint identity (where we modify operator to include fission). To see how this works, we'll take the identity we used before and let $\zeta = \delta\psi$ and $\zeta^{\dagger} = \psi_0^{\dagger}$. We can make this choice because these are functions in the function space (so it's valid) and it turns out this choice will be strategically useful. We can see:
\begin{align*}
\langle\zeta^{\dagger}, H \zeta\rangle &= \langle\zeta, H^{\dagger} \zeta^{\dagger}\rangle \\
%
\langle \psi_0^{\dagger}, \bigl(H_0 - \frac{1}{k_0}G_0 \bigr)\delta \psi \rangle &= \langle \delta \psi,  \bigl(H_0^{\dagger} - \frac{1}{k_0^{\dagger}}G_0^{\dagger} \bigr)\psi_0^{\dagger} \rangle
\end{align*}
We know that the second part of the RHS is $0$ because of the definition of the adjoint equation.\\
This means the LHS is also $0$, so we can can remove it from Eqn.~\ref{eq:perturb}.

With all of this, we can now say
\[
\frac{\delta k}{k_0^2} \approx -\frac{\langle \psi_0^{\dagger},\bigl(\delta H - \frac{1}{k_0}\delta G\bigr)\psi_0 \rangle}{\langle \psi_0^{\dagger}, G_0\psi_0 \rangle}
\]
This means that if we know the two unperturbed solutions $\psi_0^{\dagger}$ and $\psi_0$ we can find the reactivity change for any perturbations -- which is a very useful outcome.

----------------------------\\
Other \textbf{Boundary Conditions} for the adjoint operator

For reflected and periodic boundaries:
\begin{itemize}
\item Take the reflected BC configuration and unfold it to the full extent (write explicitly in each region).
\item Then, take the adjoint of this with zero outgoing flux at the boundaries.
\item Apply the same argument as in the forward case, but now we know what's going in instead of what's going out. This preserves the form of reflected and periodic BCs.
\end{itemize}

For a fixed incoming flux:
\begin{align*}
\psi(\vec{r}_s, \vOmega, E) &= \psi_{in}(\vec{r}, \vOmega, E) \quad \hat{n} \cdot \vOmega < 0 \\
\psi^{\dagger}(\vec{r}_s, \vOmega, E) &= \psi^{\dagger}_{out}(\vec{r}, \vOmega, E) \quad \hat{n} \cdot \vOmega > 0 
\end{align*}
Now, we take the inner product of the boundary conditions with the operator as the angle-normal dot product:
\begin{align*}
\int_{\Gamma} dS \int_0^{\infty} dE \bigl[&\int_{\hat{n} \cdot \vOmega < 0} d\vOmega \: \hat{n} \cdot \vOmega \psi_{in}(\vec{r}_s, \vOmega, E) \psi_{out}^{\dagger}(\vec{r}_s, \vOmega, E) +\\ 
&\int_{\hat{n} \cdot \vOmega > 0} d\vOmega \: \hat{n} \cdot \vOmega \psi_{in}(\vec{r}_s, \vOmega, E) \psi_{out}^{\dagger}(\vec{r}_s, \vOmega, E) \bigr] 
\end{align*}
We need this to vanish, thus
\begin{align*}
\int_{\Gamma} dS \int_0^{\infty} dE \int_{\hat{n} \cdot \vOmega > 0} d\vOmega \: &\bigl[-|\hat{n} \cdot \vOmega| \psi_{in}(\vec{r}_s, |\vOmega|, E) \psi_{out}^{\dagger}(\vec{r}_s, |\vOmega|, E) +\\
&\hat{n} \cdot \vOmega \psi_{in}(\vec{r}_s, \vOmega, E) \psi_{out}^{\dagger}(\vec{r}_s, \vOmega, E) \bigr] =0
\end{align*}
Some notes on this:
\begin{itemize}
\item The boundary condition  has an integral nature for $\psi_{out}^{\dagger}$.
\item This balance requires both $\psi_{out}^{\dagger}$ and $\psi_{in}$. This means we have to solve both the forward and adjoint TEs to determine these values and therefore the BCs.
\item the contribution of outgoing ``adjoint neutrons" to detector response must equal the contribution of incoming ``forward neutrons": $R = \langle \psi^{\dagger}, q_{ex} \rangle$.
\end{itemize}

\end{document}
