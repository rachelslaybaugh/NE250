\documentclass[12pt]{article}
\usepackage[top=1in, bottom=1in, left=1in, right=1in]{geometry}

\usepackage{setspace}
\onehalfspacing

\usepackage{amssymb}
%% The amsthm package provides extended theorem environments
\usepackage{amsthm}
\usepackage{epsfig}
\usepackage{times}
\renewcommand{\ttdefault}{cmtt}
\usepackage{amsmath}
\usepackage{graphicx} % for graphics files
\usepackage{tabu}

% Draw figures yourself
\usepackage{tikz} 

% writing elements
\usepackage{mhchem}

% The float package HAS to load before hyperref
\usepackage{float} % for psuedocode formatting
\usepackage{xspace}

% from Denovo Methods Manual
\usepackage{mathrsfs}
\usepackage[mathcal]{euscript}
\usepackage{color}
\usepackage{array}

\usepackage[pdftex]{hyperref}
\usepackage[parfill]{parskip}

% math syntax
\newcommand{\nth}{n\ensuremath{^{\text{th}}} }
\newcommand{\ve}[1]{\ensuremath{\mathbf{#1}}}
\newcommand{\Macro}{\ensuremath{\Sigma}}
\newcommand{\rvec}{\ensuremath{\vec{r}}}
\newcommand{\vecr}{\ensuremath{\vec{r}}}
\newcommand{\omvec}{\ensuremath{\hat{\Omega}}}
\newcommand{\vOmega}{\ensuremath{\hat{\Omega}}}
\newcommand{\sigs}{\ensuremath{\Sigma_s(\rvec,E'\rightarrow E,\omvec'\rightarrow\omvec)}}
\newcommand{\el}{\ensuremath{\ell}}
\newcommand{\sigso}{\ensuremath{\Sigma_{s,0}}}
\newcommand{\sigsi}{\ensuremath{\Sigma_{s,1}}}
%---------------------------------------------------------------------------
%---------------------------------------------------------------------------
\begin{document}
\begin{center}
{\bf NE 250, F15\\
November 4, 2015 
}
\end{center}

Last time we derived the multigroup TE using a separability assumption and discussed some of the difficulties and procedures for obtaining group constants. We got this equation
\begin{align*}
[\vOmega \cdot \nabla + \Sigma_{tg}(\vec{r})]\psi_g(\vec{r}, \vOmega) =&  \sum_{g'=1}^G \int_{4 \pi} d\vOmega'\: \Sigma_{s,gg'}(\vecr, \vOmega' \cdot \vOmega) \psi_{g'}(\vec{r}, \vOmega')\\
&+\frac{\chi_g}{4 \pi}\sum_{g'=1}^G \nu\Sigma_{fg'}(\vec{r}) \phi_{g'}(\vec{r}) + q_g(\vec{r}, \vOmega)
\end{align*}
Today we'll start with a different derivation of the MG equation.

----------------------------------------------\\
\textbf{Alternative Derivation} 

We can remove the requirement of separability in energy, which allows us to define better multigroup constants (we don't need the energy weighting function). \\
To facilitate this, we can use the \textit{Legendre expansion technique} as we did before, which allows for better weighting of the multigroup cross sections.

Recall the Legendre addition theorem and flux moment definition:
\begin{align*}
P_l(\vOmega' \cdot \vOmega) &= \frac{1}{2l+1}\sum_{m=-l}^l Y^*_{lm}(\vOmega)Y_{lm}(\vOmega')\\
\phi_{l}^{m}(\vec{r},E') &= \int_{4 \pi} d\vOmega'\: Y_{lm}(\vOmega') \psi(\vec{r}, \vOmega', E') \\
\psi(\vec{r}, \vOmega', E') &= \sum_{l=0}^{\infty} \sum_{m=-l}^l Y^*_{lm}(\vOmega)\phi_{l}^{m}(\vec{r},E')\\
\Sigma_s(E'\rightarrow E, \vOmega' \cdot \vOmega) &= \sum_{l=0}^{\infty} (2l+1) \Sigma_{s,l}(E'\rightarrow E) P_l(\vOmega' \cdot \vOmega)
\end{align*}
We will apply this in our TE 
%
\begin{align*}
\bigl[\vOmega \cdot \nabla + \Sigma_t\bigr] \psi(\vec{r}, E, \vOmega) &= \frac{\chi(E)}{4 \pi}\int_0^{\infty} dE' \: \nu(E') \Sigma_f(E') \int_{4 \pi} d\vOmega' \:\psi(\vec{r}, E', \vOmega') + q(\vec{r}, E, \vOmega)\\
 &+\sum_{l=0}^{\infty} \sum_{m=-l}^l Y^*_{lm}(\vOmega)\int_0^{\infty} dE' \Sigma_{s,l}(E'\rightarrow E) \phi_{l}^{m}(\vec{r},E')
\end{align*}
%
and integrate over energy to get groups (note that we don't change anything compared to before with the streaming term or the external source).
%
\begin{itemize}
\item Scattering:\\
We can rewrite the scattering source term
\[
q_{s,gg'} = \sum_{l=0}^{\infty} \sum_{m=-l}^l Y^*_{lm}(\vOmega)\int_{E_g}^{E_{g-1}} dE \int_{E_g'}^{E_{g'-1}} dE' \: \Sigma_{s,l}(E'\rightarrow E)\phi_{l}^{m}(\vec{r},E')
\]
We can use the angular flux moments to define a group scattering cross section moment:
\[
\Sigma_{sl,gg'} \approx \dfrac{\int_{E_g}^{E_{g-1}} dE \int_{E_g'}^{E_{g'-1}} dE' \: \Sigma_{s,l}(E'\rightarrow E)\phi_{l}^{m}(\vec{r},E')}{\phi_{l,g'}^{m}(\vec{r})}
\]
Finally
\begin{align*}
q_{s,gg'} &= \sum_{l=0}^{\infty} \sum_{m=-l}^l Y^*_{lm}(\vOmega)\Sigma_{sl,gg'}\phi_{l,g'}^{m}(\vec{r})\\
q_{s,g} &= \sum_{l=0}^{\infty} \sum_{m=-l}^l Y^*_{lm}(\vOmega)\sum_{g'=1}^G \Sigma_{sl,gg'}\phi_{l,g'}^{m}(\vec{r})
\end{align*}

\item Total interaction:\\
Our first thought is to define the group total cross section this way
\[
\Sigma_{tg}(\vec{r}, \vOmega) = \dfrac{\int_{E_g}^{E_{g-1}} dE\: \Sigma_t(\vec{r}, E) \psi(\vec{r}, \vOmega, E)}{\psi_g(\vec{r}, \vOmega)}
\]
but this adds angular dependence to the total cross section, and most codes and data aren't set up to handle this. \\
Instead, We can use the Legendre expansion of the flux and define total cross section moments as well.
\[
\int_{E_g}^{E_{g-1}} dE\: \Sigma_t(\vec{r}, E)\sum_{l=0}^{\infty} \sum_{m=-l}^l Y^*_{lm}(\vOmega)\phi_{l}^{m}(\vec{r},E) = \sum_{l=0}^{\infty} \sum_{m=-l}^l Y^*_{lm}(\vOmega)\int_{E_g}^{E_{g-1}} dE\: \Sigma_{t}(\vec{r},E)\phi_{l}^{m}(\vec{r},E)
\]
Thus, we define a total cross section moment:
\[
\Sigma_{tl,g}(\vec{r}) = \dfrac{\int_{E_g}^{E_{g-1}} dE\: \Sigma_t(\vec{r}, E) \phi_l^m(\vec{r}, \vOmega, E)}{\phi_{l,g}^m(\vec{r}, \vOmega)}
\]
and the whole term becomes
\[
= \sum_{l=0}^{\infty} \sum_{m=-l}^l Y^*_{lm}(\vOmega)\Sigma_{tl,g}(\vec{r})\phi_{l,g}^{m}(\vec{r})
\]

\item Fission remains pretty easy since it already only requires the scalar flux. 
\[
\nu\Sigma_{fg'} = \dfrac{\int_{E_g'}^{E_{g'-1}} dE'\: \nu(E') \Sigma_f(E')\phi(\vec{r}, E')}{\phi_{g'}(\vec{r})}
\]
\end{itemize}

We can combine all of this, and we add another $\Sigma_g \psi_g$ to both sides:
\begin{align*}
[\vOmega \cdot \nabla &+ \Sigma_{tg}(\vec{r})]\psi_g(\vec{r}, \vOmega) =  \frac{\chi_g}{4 \pi}\sum_{g'=1}^G \nu\Sigma_{fg'}(\vec{r}) \phi_{g'}(\vec{r}) + q_g(\vec{r}, \vOmega)\\
&+ \sum_{l=0}^{\infty} \sum_{m=-l}^l Y^*_{lm}(\vOmega)\sum_{g'=1}^G \bigl[\Sigma_{sl,gg'} + \bigl(\underbrace{\Sigma_{tg}(\vec{r})}_{\text{balances rhs}} - \Sigma_{tl,g}(\vec{r}) \bigr) \delta_{gg'}  \bigr]\phi_{l,g'}^{m}(\vec{r})
\end{align*}
%
This form looks just like the first version we derived, but the scattering term is a little wonky. \\
Note that we still have a problem. We added the $\Sigma_{tg}(\vec{r})$ to make the equation look right, but haven't explained what it is. \\
If we just weight it with the scalar flux ($P_0$ version), we get
\[
\Sigma_{tg}(\vec{r}) = \Sigma_{tg,0}(\vec{r}) = \dfrac{\int_{E_g}^{E_{g-1}} dE\:\Sigma_{t}(\vec{r}, E) \phi(\vec{r}, E)}{\phi_g(\vec{r})}
\]
This is called the ``consistent $P_N$ approximation". 

A more elegant technique uses the observation that for scattering, we usually truncate the expansion at some $l=L$. \\
We can then choose $\Sigma_{tg}(\vec{r})$ to make the $l=L+1$th component to be small. That is
\[
\sum_{m=-(L+1)}^{L+1} Y^*_{(L+1)m}(\vOmega)\sum_{g'=1}^G \bigl[\Sigma_{s(L+1),gg'} + \bigl(\Sigma_{tg}(\vec{r}) - \Sigma_{t(L+1),g}(\vec{r}) \bigr) \delta_{gg'}  \bigr]\phi_{(L+1),g'}^{m}(\vec{r}) \approx 0\:.
\]
For reactors, it is usually the case that scattering into a given group is equal to scattering out of it for most energy groups (which ones wouldn't this be true for?).\\
We can then say
\begin{align*}
\sum_{g'=1}^G \Sigma_{s(L+1),gg'}\phi_{(L+1),g'}^{m}(\vec{r}) &\approx \sum_{g'=1}^G \Sigma_{s(L+1),g'g}\phi_{(L+1),g'}^{m}(\vec{r})\\
\Sigma_{tg}(\vec{r}) &= \Sigma_{t(L+1),g}(\vec{r}) - \sum_{g'=1}^G \Sigma_{s(L+1),g'g}
\end{align*}
%
This definition is called the ``extended transport approximation".

\vspace*{1 em}
----------------------------------------------\\
\textbf{Discretization of Angle}\\
We briefly mentioned the \textbf{discrete ordinates} ($S_N$) method in mid-September, and we are going to go back over it in the context of the transport equation.\\
We'll look at the 1-D, slab version of the TE within each group (that is, we have one separate transport equation for each energy group--this is called a within group equation).
\[
\mu \frac{\partial \psi}{\partial x} + \Sigma_t(x)\psi(x,\mu) = \sum_{l=0}^L (2l+1) \Sigma_{s,l}(x) P_l(\mu)\phi_l(x) + S(x,\mu)
\]
The idea of discrete ordinates approximation is that the TE is only valid along a selected set of angles $\mu_n$, and we apply a compatible quadrature approximation to the integral term.
\begin{align*}
\mu_n \frac{\partial \psi_n}{\partial x} &+ \Sigma_t(x)\psi_n(x,\mu_n) = \sum_{l=0}^L (2l+1) \Sigma_{s,l}(x) P_l(\mu_n)\phi_l(x) + S(x,\mu_n)\\
\psi_n(x) &= \psi(x,\mu_n)\\
\phi(x) &=  \frac{1}{2}\sum_{n=1}^N w_n \psi_n(x)\\
\phi_l(x) &= \frac{1}{2}\sum_{n=1}^N w_n P_l(\mu_n)\psi_n(x)\\
w_n > 0& \quad \sum_n w_n = 2
\end{align*}
The collection of $\mu_n, w_n$ is known as the angular quadrature set. \\
We frequently select $N$ is even and $\mu_n$ are symmetric about $\mu=0$.
%
\begin{align*}
\mu_{N+1-n} &= -\mu_n\:, \quad n = 1, 2, \dots, \frac{N}{2} \quad \mu_n > 0\\
w_n &= w_{N+1-n}
\end{align*}
%
If $N$ weren't even we would get $\mu_{(N+1)/2} = 0$, which makes the derivative term disappear. It can also make our BCs ambiguous.\\
With these conditions we get advantages in the boundary conditions as well.
%
\begin{align*}
\text{reflecting at } x=0 \quad &\psi_n(0) = \psi_{N+1-n}(0)\:, \quad n = 1, 2, \dots, \frac{N}{2}\\
\text{vacuum on right at } x=a \quad &\psi_n(a) = 0\:, \quad n = \frac{N}{2}+1, \frac{N}{2}+2, \dots, N
\end{align*}
%
We have a fair bit of freedom selecting an ordinate set. \\
In slab, Legendre $P_N$ and Double Legendre $DP_N$ are the most popular (the most well known in 3D is the level symmetric). \\
We started the idea of $P_N$ in our derivation of the diffusion equation. For an expanded investigation of this and $DP_N$ see Appendix D of L\&M. \\
Another method of interest is Simplified $P_N$, $SP_N$. We won't cover this, but I'll post a resource about it on our website for anyone who is curious.

This DO formulation
\begin{itemize}
\item gives us $N$ ODEs and $N$ BCs
\item treats the BCs exactly as long as the direction is in the quadrature set
\item the scattering term is an approximation. You have to interpolate if you need $\psi$ along a non-included angle.
\end{itemize}
 



\end{document}
