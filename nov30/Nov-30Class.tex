\documentclass[12pt]{article}
\usepackage[top=1in, bottom=1in, left=1in, right=1in]{geometry}

\usepackage{setspace}
\onehalfspacing

\usepackage{amssymb}
%% The amsthm package provides extended theorem environments
\usepackage{amsthm}
\usepackage{epsfig}
\usepackage{times}
\renewcommand{\ttdefault}{cmtt}
\usepackage{amsmath}
\usepackage{graphicx} % for graphics files
\usepackage{tabu}

% Draw figures yourself
\usepackage{tikz} 
\usetikzlibrary{shapes,arrows}

% writing elements
\usepackage{mhchem}

% The float package HAS to load before hyperref
\usepackage{float} % for psuedocode formatting
\usepackage{xspace}

% from Denovo Methods Manual
\usepackage{mathrsfs}
\usepackage[mathcal]{euscript}
\usepackage{color}
\usepackage{array}

\usepackage[pdftex]{hyperref}
\usepackage[parfill]{parskip}

% math syntax
\newcommand{\nth}{n\ensuremath{^{\text{th}}} }
\newcommand{\ve}[1]{\ensuremath{\mathbf{#1}}}
\newcommand{\Macro}{\ensuremath{\Sigma}}
\newcommand{\rvec}{\ensuremath{\vec{r}}}
\newcommand{\vecr}{\ensuremath{\vec{r}}}
\newcommand{\omvec}{\ensuremath{\hat{\Omega}}}
\newcommand{\vOmega}{\ensuremath{\hat{\Omega}}}
\newcommand{\sigs}{\ensuremath{\Sigma_s(\rvec,E'\rightarrow E,\omvec'\rightarrow\omvec)}}
\newcommand{\el}{\ensuremath{\ell}}
\newcommand{\sigso}{\ensuremath{\Sigma_{s,0}}}
\newcommand{\sigsi}{\ensuremath{\Sigma_{s,1}}}
\newcommand{\ep}{\ensuremath{\varepsilon}}
%---------------------------------------------------------------------------
%---------------------------------------------------------------------------
\begin{document}
\begin{center}
{\bf NE 250, F15\\
November 30, 2015 
}
\end{center}

We've talked about how to discretize the transport equation in energy using the \textit{multigroup approximation,} in angle using $S_N$, $P_N$, or $SP_N$, and in space using \textit{diamond difference} or an equivalent spatial scheme. \\
If we're solving an eigenvalue problem, we also need to deal with that.\\
Great. Now how to we actually solve it all? 

Well, we break this into two iteration levels if there isn't fission and three iteration levels if there is fission:
\begin{enumerate}
\item \textbf{inner iterations}: solve the space-angle component for each energy group. This uses the sweeps that we talked about last time. \\
We get to choose what kind of solver we would like to use. \\
The most basic and traditional is source iteration; a more advanced choice is a Krylov solver.

\item \textbf{outer iterations}: solve the energy component; after we've dealt with space and angle in each group, we iterate on energy if necessary. \\
The most common solver is Gauss Seidel, with Jacobi or block Jacobi being simpler and multigroup Krylov solves being a new option.

\item \textbf{eigenvalue iterations}: after the space, angle, and energy has been converged, we update the eigenvalue and eigenvector until they converge.\\
The historic solver is Power Iteration.\\
Other choices are shifted inverse iteration (Weilandt's method and Rayleigh Quotient Iteration are extensions of this), Arnoldi, and Davidson.
\end{enumerate}
At each step each solver has pros and cons. We think about things like convergence rate, ability to parallelize, and ability to precondition/accelerate.\\
Each iteration level often comes with its own set of preconditioners and/or acceleration strategies.

\textit{Inner Iterations:}\\
Inner iteration index is $p$: we need to know $\phi_{l,i}^{(p-1)}$, so for $p=1$ we have an initial guess for $\phi_{l,i}^{(0)}$. We use this strategy:

% Define block styles
\tikzstyle{decision} = [diamond, draw,% fill=blue!20, 
    text width=5em, text badly centered, node distance=3cm, inner sep=0pt]
\tikzstyle{block} = [rectangle, draw,% fill=blue!20, 
    text width=10em, text centered, rounded corners, minimum height=4em]
\tikzstyle{line} = [draw, -latex']
\tikzstyle{cloud} = [draw, ellipse, node distance=3cm, %fill=red!20, 
    minimum height=2em]
%
\begin{center}
\begin{tikzpicture}[node distance = 2cm, auto]
    % Place nodes
    \node [block] (init) {Angular\\moments, $\phi_{l,i}^{(p-1)}$};   
    \node [cloud, right of=init, node distance=5cm] (input) {$\phi_{l,i}^{(0)}$};
    \node [block, below of=init, node distance=2.5cm] (source) {scattering formula + external source: $q_{n,1}^{(p-1)}$};
    \node [block, below of=source, node distance=3cm] (sweep) {angular flux, $\psi_{n,i}^{(p)}$};
        \node [block, below of=sweep, node distance=2.5cm] (moments) {update angular flux moments, $\phi_{l,i}^{(p)}$};
    \node [decision, below of=moments, node distance=3cm] (converged) {$\phi_{0,i}$ converged?};
    \node [block, right of=converged, node distance=5cm] (stop) {terminate successfully:\\ solution is $\phi_{l,i}^{(p)}$};
    \node [decision, left of=converged, node distance=5cm] (max) {$p \geq p_{\max}$?};
    \node [block, left of=sweep, node distance=5cm] (update) {$\phi_{l,i}^{(p-1)}= \phi_{l,i}^{(p)}$\\$p = p+1$};
    \node [block, below of=max, node distance=3cm] (end) {terminate unsuccessfully};
    % Draw edges
    \path [line] (init) -- (source);
    \path [line] (source) -- node {perform mesh sweep}(sweep);
    \path [line] (sweep) -- (moments);    
    \path [line] (moments) -- (converged);
    \path [line] (converged) -- node {no}(max);
    \path [line] (converged) -- node {yes}(stop);    
    \path [line] (max) -- node {yes}(end);    
    \path [line] (max) -- node {no}(update);    
    \path [line] (update) |- (init);
    \path [line,dashed] (input) -- (init);
\end{tikzpicture}
\end{center}

\end{document}